\documentclass{beamer}

\usepackage[utf8]{inputenc}
\usepackage[spanish, mexico]{babel}

\title{La Función Beta}
\author{Miguel Angel Gomez Barrera}
\institute{Fundación Universitaria Konrad Lorenz}
\date{2020}

\begin{document}
	\frame{\titlepage}
	\begin{frame}
	\frametitle{Preliminares}
		La función Beta y la función Gamma son conocidas como las integrales de Euler y se encuentran clasificadas en dos tipos:
		\begin{itemize}
			\item Del primer tipo: la función Beta.
			\item Del segundo tipo:la función Gamma.
			$$\Gamma(x) = \int_{0}^{\infty} t^{x-1} e^{-t} dt,$$
			con $x \in \mathbb{C}-\mathbb{Z}^-$, ya hemos visto que esta función contiene la función factorial, de modo que:
			$$\Gamma(x+1) = (x-1)! = \int_{0}^{\infty} t^{x-1} e^{-t} dt$$
		\end{itemize}		
	\end{frame}
	\begin{frame}
		\frametitle{La Función Beta.}
		Conocida también como la integral de primer tipo de Euler, es de gran importancia debido a su conexión con la función gamma y a que muchas integrales complejas pueden ser reducidas a expresiones que involucran a ésta función.
		\begin{block}{Definición}
			La función beta, se denota como $B(x,y)$, definida como
			$$B(x,y) = \int_{0}^{1} t^{x-1}(1-t)^{y-1}dt$$
		\end{block} 
	\end{frame}
	\begin{frame}
		\frametitle{La Función Beta.}
		\begin{block}{Propiedad de simetría.}
			$$B(x,y) = B(y,x).$$
			\begin{proof}
				Por la propiedad de convergencia de las integrales definidas tenemos
				$$\int_{0}^{a} f(t) dt = \int_{0}^{a} f(a-t) dt,$$
				De modo que rescribiendo la integral de la definición, tenemos
				$$B(x,y) = \int_{0}^{1} t^{y-1}(1-t)^{x-1} dt,$$
				vemos por lo tanto que $B(x,y) = B(y,x)$. 
			\end{proof}
		\end{block}
	\end{frame}
	\begin{frame}
	\frametitle{La Función Beta.}
		\begin{block}{Relación con la función Gamma.}
			Tenemos que
			$$B(x,y) = \frac{\Gamma(x)\Gamma(y)}{\Gamma(x+y)},$$
			para todos los enteros positivos $x$ y $y$, definimos la función beta como
			$$B(x,y) = \frac{(x-1)!(y-1)!}{(x+y-1)!}$$
		\end{block}
	\end{frame}
	\begin{frame}
	\frametitle{La Función Beta.}
	\begin{block}{Relación con la función Gamma.}
		\begin{proof}
			Por la definición de la función gamma
			$$\Gamma(s) = \int_{0}^{\infty} x^{s-1}e^{-x} dx,$$
			podemos escribir
			$$\Gamma(m)\Gamma(n) = \int_{0}^{\infty} x^{m-1}e^{-x} dx \int_{0}^{\infty} y^{n-1}e^{-y} dy,$$
			y que podemos rescribir como la integral doble
			$$\Gamma(m)\Gamma(n) = \int_{0}^{\infty}\int_{0}^{\infty} x^{m-1} y^{n-1}e^{-x-y} dx dy,$$ 
		\end{proof}
	\end{block}
	\end{frame}
	\begin{frame}
	\frametitle{La Función Beta.}
	\begin{block}{Relación con la función Gamma.}
		\begin{proof}
			$$\Gamma(m)\Gamma(n) = \int_{0}^{\infty}\int_{0}^{\infty} x^{m-1} y^{n-1}e^{-x-y} dy,$$
			Aplicamos la sustitución $x = vt$ y $y = v(1-t)$, tenemos
			$$\Gamma(m)\Gamma(n) = \int_{0}^{1} t^{m-1}(1-t)^{n-1} dt \int_{0}^{\infty} v^{m+n-1}e^{-v} dv,$$
			que es exactamente la definición de la función gamma y la función Beta, luego
			$$\Gamma(m)\Gamma(n) = B(m,n)\Gamma(m+n)$$
			y por tanto
			$$B(m,n) = \frac{\Gamma(m)\Gamma(n)}{\Gamma(m+n)}$$
		\end{proof}
	\end{block}
	\end{frame}
\end{document}