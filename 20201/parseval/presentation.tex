\documentclass{beamer}

\usepackage[utf8]{inputenc}
\usepackage[spanish, mexico]{babel}

\title{Identidad de Parseval}
\author{Miguel Angel Gomez Barrera}
\institute{Fundación Universitaria Konrad Lorenz}
\date{2020}

\begin{document}
	\frame{\titlepage}
	\begin{frame}
	\frametitle{Resultados preliminares}
	Sea número complejo $z = a + ib$, su conjugado se le de denota por $z^*$ o $\bar{z}$ tal que  $\bar{z} = a - ib$. \pause
	\begin{block}{Propiedad producto}
		$$z \cdot \bar{z} = a^2 + b^2$$
	\end{block}
	\end{frame}
	\begin{frame}
		\frametitle{Resultados preliminares}
		\framesubtitle{Serie compleja de Fourier.}
		En este momento tenemos que una serie de Fourier tiene la forma:
		$$f(x) = a_0 + \sum_{n=1}^{\infty} a_n \cos(nx) + \sum_{n=1}^{\infty} b_n \sin(nx),$$
		\pause
		también tenemos que las funciones trigonométricas las podemos expresar como:
		$$\cos(x) = \frac{e^{ix} + e^{-ix}}{2} \text{, y que } \sin(x) = \frac{e^{ix} - e^{-ix}}{2i} = \frac{-ie^{ix} + ie^{-ix}}{2},$$
		\pause
		Reemplazando esto en la serie original, tenemos 
		$$ a_0 + \sum_{n=1}^{\infty} a_n \left( \frac{e^{inx} + e^{-inx}}{2} \right) + \sum_{n=1}^{\infty} b_n \left(\frac{-ie^{inx} + ie^{-inx}}{2}\right),$$
	\end{frame}
	\begin{frame}
	\frametitle{Resultados preliminares}
	\framesubtitle{Serie compleja de Fourier.}
	Agrupando términos de la serie obtenemos:
	$$a_0 + \sum_{n=1}^{\infty} \frac{a_n - ib_n}{2} e^{inx} + \sum_{n=1}^{\infty} \frac{a_n + ib_n}{2} e^{-inx},$$
	Podemos cambiar los índices de las las sumas de $n$ a $-n$, al efectuar este cambio podemos rescribir la expresión anterior en:
	$$ f(x) = \sum_{n = -\infty}^{\infty} c_n e^{inx},$$
	donde $c_n = \frac{a_n - ib_n}{2}$, ahora debemos hallar una fórmula para hallar $c_n$, para ello diremos que:
	$$f(x)e^{-imx} = \sum_{-\infty}^{\infty} c_n e^{inx} e^{-imx},$$
	\end{frame}
	\begin{frame}
	\frametitle{Resultados preliminares}
	\framesubtitle{Serie compleja de Fourier.}
	Agrupando términos de la serie obtenemos:
	$$a_0 + \sum_{n=1}^{\infty} \frac{a_n - ib_n}{2} e^{inx} + \sum_{n=1}^{\infty} \frac{a_n + ib_n}{2} e^{-inx},$$
	Podemos cambiar los índices de las las sumas de $n$ a $-n$, al efectuar este cambio podemos rescribir la expresión anterior en:
	$$ f(x) = \sum_{n = -\infty}^{\infty} c_n e^{inx},$$
	donde $c_n = \frac{a_n - ib_n}{2}$, ahora debemos hallar una fórmula para hallar $c_n$, para ello diremos que:
	$$f(x)e^{-imx} = \sum_{-\infty}^{\infty} c_n e^{inx} e^{-imx},$$
	\end{frame}
	\begin{frame}
	\framesubtitle{Serie compleja de Fourier.}
	E integrando entre $-\pi$ y $\pi$:
	$$\int^{\pi}_{-\pi} f(x)e^{-imx} dx= \sum_{-\infty}^{\infty} c_n \int^{\pi}_{-\pi}e^{inx} e^{-imx} dx = 2\pi c_n,$$
	luego,
	$$c_n = \frac{1}{2\pi}\int_{-\pi}^{\pi} f(x) e^{-inx} dx$$
	\end{frame}
	\begin{frame}
	\frametitle{Identidad de Parseval}
	Si $X(t)$ es la serie compleja de Fourier, con período $T_0$, la función cuadrado integrable $P(x)$ satisface:
	$$P(x) = \sum_{n=-\infty}^{\infty} |c_n|^2$$
	\textit{Prueba.} 
	$$X(t) = \sum_{-\infty}^{\infty} c_n e^{inw_0t},$$
	por lo tanto su conjugada será,
	$$\bar{X}(t) = \sum_{-\infty}^{\infty} c_n e^{-inw_0t},$$
	si multiplicamos ambos obtendremos por la propiedad de complejos que:
	$$X(t)\cdot\bar{X}(t) = |X(t)|^2,$$
	\end{frame}
	\begin{frame}
	\frametitle{Identidad de Parseval}
	\begin{align*}
		P(x) &= \frac{1}{T_0} \int_{0}^{T_0} |X(t)|^2 dt\\
		&= \frac{1}{T_0} \int_{0}^{T_0} X(t)\cdot\bar{X}(t)\\
		&= \frac{1}{T_0} \int_{0}^{T_0} X(t)\cdot\sum_{n=-\infty}^{\infty} \bar{c}_n e^{-inw_0t} dt\\
		&= \sum_{n=-\infty}^{\infty} \bar{c}_n \frac{1}{T_0} \int_{0}^{T_0} X(t)\cdot e^{-inw_0t} dt\\
	\end{align*}
	nótese que el término $\frac{1}{T_0} \int_{0}^{T_0} X(t)\cdot e^{-inw_0t} dt$ es equivalente a $c_n$, por ende
	$$ =  \sum_{n=-\infty}^{\infty} \bar{c}_n \cdot c_n = \sum_{n=-\infty}^{\infty} |c_n|^2$$
	\end{frame}
\end{document}